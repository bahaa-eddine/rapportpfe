\chapter*{R\'esum\'e}

Dans le cadre de ce Projet de Fin d'\'etudes, nous \'etions responsables de la conception des diff\'erentes fonctionnalit\'es d'une solution moderne Prospektor. Cette solution vient dans un contexte de digitalisation du processus de gestion des anomalies dans les mines du groupe \gls{OCP} lors de l'extraction de phosphates en se basant sur une architecture orient\'ee \gls{API}. Notre contribution dans ce projet a commenc\'e par une analyse de l'existant pour pr\'eciser le point de d\'epart, ensuite nous avons commenc\'e \`a impl\'ementer les fonctionnalit\'es par Epics dans une approche Agile.

Ce stage de fin d'\'etude nous a \'et\'e tr\`es b\'en\'efique, il nous a aid\'e \`a projeter l'ensemble des connaissances acquises durant notre formation d'ing\'enieur \`a l'ENSA de Khouribga, ainsi que l'implication dans les diff\'erentes phases li\'ees \`a un projet de d\'eveloppement logiciel. Une forte valeur ajout\'ee r\'eside dans l'apprentissage des nouveaux concepts qui sont actuellement la tendance du domaine du g\'enie logiciel \`a savoir l'utilisation d'une architecture orient\'ee \gls{API}s bas\'e sur les web service qui tient en compte le contexte actuel de la technologie ainsi que l'adoption d'une m\'ethodologie agile qui nous a permis une marge de flexibilit\'e pour innover et livrer plus de valeur. Encore plus, cette exp\'erience de stage nous a aid\'e \`a int\'egrer le monde de l'entreprise, l'interaction et le travail au sein d'une \'equipe qui ont rendu ce projet une exp\'erience professionnelle tr\`es r\'eussie.

La solution Prospektor n'est pas encore termin\'ee en terme de d\'eveloppement vu la voluminosit\'e de l'ensemble des fonctionnalit\'es. Il reste encore à d\'evelopper des fonctionnalit\'es en relation avec le reste des Epics de la RoadMap ainsi que les t\^aches li\'ees aux analytiques.

\vspace{1.5\baselineskip}

\textbf{Mots cl\'es} : Prospektor, Orient\'e API, Aproche Agile.