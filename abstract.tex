\chapter*{Abstract}

As part of the graduation project, we were responsible for designing the various functionalities of a modern prospektor solution. This solution is part of the digitalization of the OCP's anomaly management process, it's based on an API oriented architecture. Our contribution in this project started with an analysis of the existing tools to specify the departure point, then we started to implement the functionalities by Epics in an Agile approach.

This graduation internship was very beneficial, it helped us to project all the knowledge acquired during our engineering studies at ENSA Khouribga, as well as the involvement in the different steps related to a software development project. A strong added value lies in learning new concepts that are currently the trend of the software engineering field namely the use of an API oriented architecture based on web services that considers the current context of technology, as well as the adoption of an agile methodology that has given us a margin of flexibility to innovate and deliver more value. Even more, this internship experience helped us to integrate the business world, interaction and working within a team that made this project a very successful work experience.

The Prospektor solution is not yet finished in its development due to the volume of all the functionalities. It still remains to develop features related to the rest of the RoadMap Epics as well as the tasks related to analytics.


\vspace{1.5\baselineskip}

\textbf{Keywords} : Prospektor, API oriented architecture, Agile methodology.