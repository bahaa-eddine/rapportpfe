\chapter{Analyse et sp\'ecification des besoins}
\section{Introduction}
Ce chapitre est consacr\'e \`a l'analyse et \`a la sp\'ecification des besoins fonctionnels et non fonctionnels de la solution qui est une \'etape primordiale pour la r\'ealisation de notre projet.
\section{Analyse des besoins}
Dans cette partie, nous pr\'esenterons les besoins fonctionnels et non fonctionnels identifi\'es apr\`es la s\'election des besoins.
\subsection{Identification des acteurs}
Dans le cas de notre projet on consid\`ere deux acteurs :
\begin{itemize}
\item \textbf{Le prospecteur} : Il a comme mission principale de g\'erer les anomalies de la plateforme.
\item \textbf{L'exploitant} : Il permet de r\'epondre sur les anomalies qui ne sont pas encore r\'esolu.
\end{itemize}
\subsection{Les besoins fonctionnels}
Au cours de cette \'etape, nous allons extraire les diff\'erentes fonctionnalit\'es offertes par notre projet.
\begin{itemize}
\item L'application prospektor doit permettre \`a chaque prospecteur de :
\begin{itemize}
\item suivre et consulter les anomalies qui a cr\'eer.
\item consulter la liste des attachements li\'ees par anomalies.
\end{itemize}
\item L'application prospektor doit permettre \`a chaque exploitant de :
\begin{itemize}
\item suivre et consulter les anomalies par \'etape, date \& criticit\'e.
\item consulter les attachements li\'ees par anomalies.
\item Ajouter des attachements aux anomalies qui ne sont pas encore r\'esolu.
\end{itemize}  
\end{itemize}

\subsection{Les besoins non fonctionnels}
Outre les fonctions cit\'ees ci-dessus, l'application doit assurer en certaine mesure les caract\'eristiques suivantes :
\begin{itemize}
\item L'efficacit\'e : L'efficacit\'e de l'application doit permettre l'accomplissement de la t\^ache avec le minimum de manipulation. Ceci doit \^etre garanti pour que l'application puisse s'int\'egrer facilement dans l'environnement ou elle va \^etre d\'eploy\'ee.

\item La s\'ecurit\'e : Les diff\'erents comptes utilis\'es par les utilisateurs doivent \^etre s\'ecuris\'es et v\'erifi\'es pour \'eviter les faux comptes et les fausses informations.

\item La fiabilit\'e : Touche \`a l'aspect qualit\'e des donn\'ees et persistance des informations dans l'application ainsi que la vitesse de chargement des interfaces.
\item La performance : le temps de r\'eponse de la plateforme doit \^etre rapide.

\item La maintenabilit\'e : La solution doit \^etre stable face aux changements, ainsi qu'un fort niveau de testabilit\'e assur\'e par les tests fonctionnels.

\item La scalabilit\'e : la solution doit d'\^etre extensible en termes de la charge des requ\^etes trait\'ees.

\item L'\'evolutivit\'e : possibilit\'e d'ajout des nouvelles fonctionnalit\'es au cours du temps selon le besoin des fournisseurs.

\item Le d\'eploiement intelligent: l'introduction des nouveaux changements ne doit pas impacter les modules existants, d'o\`u le besoin d'une d\'emarche de d\'eploiement intelligente de chaque module.

\item La portabilit\'e: facilit\'e de passage d'un environnement de d\'eveloppement et tests vers un environnement de pr\'e-production ou un environnement de production.
\end{itemize} 


\subsection{Diagramme des cas d'utilisation}
Cette figure repr\'esente le diagramme de cas d'utilisation globale de l'acteur prospecteur :
\begin{figure}[H]
	\center{\includegraphics[width=\textwidth]{Figures/prospectorDiagram.png}}
	\caption{\label{fig:my-label} Diagramme de cas d'utilisation globale pour le prospecteur}
\end{figure}

Cette figure repr\'esente le diagramme de cas d'utilisation globale de l'acteur exploitant :
\begin{figure}[H]
	\center{\includegraphics[width=\textwidth]{Figures/exploitantDiagram.png}}
	\caption{\label{fig:my-label} Diagramme de cas d'utilisation globale pour l'exploitant}
\end{figure}

\subsection{Description des cas d'utilisation}
\textbf{Prospecteur :} Les grandes \'etapes pour un prospecteur lors de l'utilisation de l'application prospektor.
\begin{itemize}
  \item Ouvrir l'application prospektor
  \item S'authentifier
  \begin{itemize}
    \item Consulter les anomalies en cours de traitement
    \item Consulter les demandes d'intervention
    \item Consulter l'historique des anomalies
    \item Consulter le rapport des tourn\'ees
    \item D\'emarrer \& Continuer une tourn\'ee
    \begin{itemize}
      \item sp\'ecifier la localisation du prospection
      \item Remplir la check-list des \'el\'ements \`a auditer
      \item Cr\'eation d'une anomalie au cas d'un d\'erangement
      \item Ajouter des attachements (audio \& photos) si n\'ecessaire
    \end{itemize}
    \item Continuer la tourn\'ee vers un autre chantier
    \item Obtenir le rapport complet de la tourn\'ee 
  \end{itemize}
\end{itemize}

\textbf{Exploitant :} Les grandes \'etapes pour un exploitant lors de l'utilisation de l'application prospektor.
\begin{itemize}
  \item Ouvrir l'application prospektor
  \item S'authentifier
  \begin{itemize}
    \item Consulter les anomalies en cours
    \item Consulter les anomalies \`a traiter
    \item Consulter l'historique des anomalies
  \end{itemize}
\end{itemize}

    


    



\section{Conclusion}