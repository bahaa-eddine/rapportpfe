\chapter{Analyse et sp\'ecification des besoins}
\section{Introduction}
Ce chapitre est consacr\'e \`a l'analyse et \`a la sp\'ecification des besoins fonctionnels et non fonctionnels de la solution qui est une \'etape primordiale pour la r\'ealisation de notre projet.
\section{Analyse des besoins}
Dans cette partie, nous pr\'esenterons les besoins fonctionnels et non fonctionnels identifi\'es apr\`es la s\'election des besoins.
\subsection{Identification des acteurs}
Dans le cas de notre projet on consid\`ere deux acteurs :
\begin{itemize}
\item \textbf{Le prospecteur} : Il a comme mission principale de g\'erer les anomalies de la plateforme.
\item \textbf{L'exploitant} : Il permet de r\'epondre sur les anomalies qui ne sont pas encore r\'esolu.
\end{itemize}
\subsection{Les besoins fonctionnels}
Au cours de cette \'etape, nous allons extraire les diff\'erentes fonctionnalit\'es offertes par notre projet.
\begin{itemize}
\item L'application prospektor doit permettre \`a chaque prospecteur de :
\begin{itemize}
\item suivre et consulter les anomalies qui a cr\'eer.
\item consulter la liste des attachements li\'ees par anomalies.
\end{itemize}
\item L'application prospektor doit permettre \`a chaque exploitant de :
\begin{itemize}
\item suivre et consulter les anomalies par \'etape, date \& criticit\'e.
\item consulter les attachements li\'ees par anomalies.
\item Ajouter des attachements aux anomalies qui ne sont pas encore r\'esolu.
\end{itemize}  
\end{itemize}



\subsection{Diagramme des cas d'utilisation}
\subsection{Description des cas d'utilisation}
\section{Conclusion}