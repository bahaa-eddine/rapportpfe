\chapter{Introduction}


La gestion des activit\'es m\'etiers devient de plus en plus un d\'efi majeur pour les soci\'et\'es, un d\'efi qui est aujourd'hui un point d\'eterminant en termes d'optimisation des processus m\'etiers ainsi que l'am\'elioration de leur visibilit\'e et de leur gestion. Les entreprises manufacturi\`eres changent de strat\'egie au fur et \`a mesure de l'\'evolution des march\'es. Soumises \`a de fortes pressions concurrentielles au cours des derni\`eres d\'ecennies, les industries se sont orient\'ees vers la digitalisation.

Dans ce cadre-l\`a s'inscrit le sujet de mon \gls{PFE} au sein de l'\gls{OCP}, dont le but est de concevoir et impl\'ementer une solution informatique avec une architecture moderne pour digitaliser le processus m\'etier. Dans notre situation est gestionn\'e les anomalies dans les sites de group \gls{OCP}.

Le point de d\'epart de notre projet est de faire une analyse profonde pour r\'ealiser la premi\`ere version \gls{MVP} ( Une r\'ealisation qu'on peut la mettre en face des clients pour commencer \`a valider nos hypoth\'eses), apr\`es on va faire des am\'eliorations correspondants aux nos besoins. L'\'equipe  travaille avec une m\'ethodologie Scrum selon les Epics trac\'es dans la RoadMap du projet.

Le pr\'esent rapport d\'ecrit l'ensemble du travail r\'ealis\'e dans le cadre de ce projet, il contient quatre chapitres. Le premier chapitre contient une description du contexte g\'en\'eral du projet notamment la pr\'esentation de la Digital Factory de l'\gls{OCP} ainsi que la motivation et les objectifs du projet. Le deuxid\`eme chapitre pr\'esente une analyse de besoins fonctionnels et non fonctionnels. Par la suite le troisid\`eme chapitre mettra l'accent sur l'ensemble des \'el\'ements de l'\'etude conceptuelle. Enfin, le chapitre quatre pr\'esentera les r\'esultats de l'impl\'ementation.