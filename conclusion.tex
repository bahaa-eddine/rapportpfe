\chapter*{Conclusion g\'en\'erale}

Ce m\'emoire pr\'esente le bilan du travail effectu\'e durant la p\'eriode de notre stage de Fin d'Etudes au sein de la Digital Factory de l'Office Ch\'erifien des Phosphates. Le but de notre stage est de contribuer \`a la conception ainsi que le d\'eveloppement d'une solution moderne pour le gestionnement et le suivi complet des anomalies dans les mines du groupe \gls{OCP}.

La naissance de ce projet est due au besoin de la centralisation des donn\'ees ainsi minimisation le temps de r\'eponse pour r\'esoudre les anomalies lors de l'extraction de phosphates. En effet, la plateforme Prospektor pr\'esente une r\'esolution \`a cette probl\'ematique en se basant sur une vision architecturale qui r\'epond aux besoins de modularit\'e ainsi que la scalabilit\'e.

Durant l'ensemble des activit\'es men\'ees dans le cadre de ce projet, nous avons pass\'e par l'ensemble des phases d'un projet logiciel ainsi que nous avons utilis\'e plusieurs technologies notamment Spring Boot comme framework d'impl\'ementation de \gls{API} Backend et Android Native \& ReactJs comme framework d'impl\'ementation de la logique Frontend. Le projet s'est d\'eroul\'e en adoptant une m\'ethodologie agile bas\'ee essentiellement sur Scrum comme processus de d\'eveloppement afin de maximiser la livraison de valeur dans un minimum de temps.

Durant ce travail, on a pu r\'ealiser plusieurs fonctionnalit\'es, \`a savoir le suivi complet des anomalies dans les sites, leurs attachements et les commentaires des utilisateurs sur ces anomalies. Toutefois, il reste plusieurs fonctionnalit\'es \`a r\'ealiser, \`a savoir stocker les donn\'ees localement en cas d'absence de connexion internet, envoyer  des notifications par \gls{SMS} ainsi les analytiques et le moteur de recherche qui doivent \^etre livr\'ees les mois qui suivent.